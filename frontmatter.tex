
\noindent This report is an account of work funded by the US Department of Energy Office of Electric's Transactive Energy Program, and managed by Christopher Irwin (christopher.irwin@hq.doe.gov).

\vspace{0.25in}

\noindent SLAC National Accelerator Laboratory is operated by Stanford University for the US Department of Energy under Contract DE-AC02-76SF00515.

\vspace{0.25in}

\noindent The principle investigator may be contacted via email at \textit{dchassin@slac.stanford.edu}.

\vspace{5in}

\noindent Copyright \textcopyright \ 2021, Regents of the Leland Stanford Junior University \\ All Rights Reserved

\newpage

\section*{Executive Summary}

    An electric utility planning and operations platform that supports the design, deployment, and operation of peer-to-peer trading allows more efficient and resilient integration of distributed energy resources in an increasingly decarbonized bulk electric power system. Distributed energy resources can contribute to decarbonization, but the main benefits are lost unless the utility sustains financial performance, consumers are engaged and committed to participating in the long term, and the bulk electric power system maintains high reliability. The Transactive Energy Services System (TESS) provides a whole-business life-cycle approach to transactive energy services using an end-to-end framework that a majority of North American utilities can use to finance, design, deploy, and operate a distributed energy resource integration system to realize these three key performance objectives. 

    This report describes the first deployment of TESS at Holy Cross Energy in Basalt, Colorado.
    
    TODO: key insights, summary of recommendations, etc.
